\documentclass[11pt]{article}
\usepackage[utf8]{inputenc}
\usepackage{graphicx}
\usepackage{hyperref}
\usepackage{biblatex}
\usepackage[margin=0.5in]{geometry}

\title{Topic 7: PostGIS}
\author{Christine E. Pilato}
\date{July 22, 2019}

\begin{document}
\begin{figure}
    \centering
    \includegraphics[width=4cm]{postgis.png}
\end{figure}
\maketitle

\section*{Choose an Open Source Project}
PostGIS is a spatial database extender for a PostgreSQL relational database.
\section*{About the Project License and Project Status}
The project license for PostGIS is under GNU General Public License (GNU GPL) version 2 or later, which is approved by the Free Software Foundation (FSF). This a copyleft license that states the end-user must satisfy the license obligation (FSF, 1991). PostGIS is still an active project \textemdash version 3.0.0 alpha3 was just released.
\section*{Reporting Bugs and Commercial Support}
\href{https://postgis.net/support/}{PostGIS.net} provides a wide variety of options for end-user support. If a end-user is specifically reporting a bug, they can take advantage of the \href{http://trac.osgeo.org/postgis}{online ticket tracker}. To ask for additional help, end-users can also use the \href{https://lists.osgeo.org/mailman/listinfo/postgis-users}{mailing list}. PostGIS additionally lists commercial resources as well.
\section*{Requesting New Features and Community Involvement}
The source code for both historical and current versions of PostGIS is hosted as an HTML, PDF, .gz files on PostGIS's website: \url{https://postgis.net/source/}. To add new features, end-users are encouraged to thoroughly document and submit their patches through this link: \url{http://trac.osgeo.org/postgis/wiki/DevWikiDocNewFeature}.

\begin{thebibliography}
\bibitem{Free} Free Software Foundation (FSF). (1991). GNU General Public License, version 2. Retrieved from \url{https://www.gnu.org/licenses/old-licenses/gpl-2.0.html}
\end{thebibliography}

\end{document}
